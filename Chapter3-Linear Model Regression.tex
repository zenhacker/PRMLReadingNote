\documentclass[paper=a4,fontsize=11pt]{xetexpNote}

\lhead{}
\rhead{}
\lfoot{}
\rfoot{~\thepage~}
\usepackage[round, comma]{natbib}

\begin{document}
\NoteTitleName{Online Object Tracking: A Benchmark}\sepspace
\MyBrief{{\Huge 本}文总结了这篇文章中关于Related Work部分的归类总结,以便以后参考。Object Tracking任务主要包括3个子任务:Target Representation Scheme (或Appearance Model)、Search Scheme、Model Update,下面就分别从这三个方面来总结各自的研究成果。}
\sepspace


%\MyPartb{\HandRight\ \ Brief}{}
\begin{compactitem}[\color{RoyalBlue}$\circ$]
	\item 首先Linear Model的含义是模型对于未知参数是线性的,而不要求对input是否线性,事实上为了通常input都是经过非线性处理的。未知参数的线性形式为分析提供了方便,当然同时也一定程度上限定了模型的能力。最简单的一个线性模型就是Linear Regression,它对于input和未知参数都是线性的。
    \item \textbf{Linear Model}:
    
    \qquad 给定一个含有$N$个数据的训练数据集$\{\mathbf{x}_n\}$(其中$n=1,...,N$),同时给定了训练数据的标签值$\{t_n\}$。
    
    \checkmark \textcolor[rgb]{1.00,0.00,0.00}{问题定义}:\[ y(\mathbf{x}, \mathbf{w}) =    \
                                                w_0 + \sum_{j=1}^{M-1} w_j \phi_j(\mathbf{x}) = \
                                                \sum_{j=0}^{M-1} w_j \phi_j(\mathbf{x}) = \                                                
                                                \mathbf{w}^T \phi(\mathbf{x}) \]
    \qquad 其中M定义了模型的尺寸或者复杂度;$\phi_j(\mathbf{x})$称为\emph{basis functions},一般是非线性的,可以看作是原始input的一种变换或者特征提取。
    
    \qquad Linear Regression中basis function选择的就是$\phi_j(\mathbf{x}) = x_j$;Polynomial Regression中选择的是$\phi_j(\mathbf{x}) = x^j$。
    
    \qquad 还有一些常用的basis functions有:
    \[ \mathrm{Gaussian\ Basis\ Function:} \phi_j(\mathbf{x}) = \exp\{ -\frac{(x-\mu_j)^2}{2s^2} \} \]
    \[ \mathrm{Sigmodial\ Basis\ Function:} \phi_j(\mathbf{x}) = \sigma( \frac{x-\mu_j}{s} ) \]
    此外,还有Fourier basis、wavelets正交基等。
      
    \checkmark \textcolor[rgb]{1.00,0.00,0.00}{解决方案}:   
    
    $\bullet$ Least Squares最小二乘法:就是最小化sum-of-squares error function(均方误差),目标函数为:
    \[ E_D(\mathbf{w}) = \frac{1}{2} \sum_{n=1}^{N}\{t_n - \mathbf{w}^T \phi(\mathbf{x}_n)\} ^ 2 \]
    
    $\bullet$ Maximum Likelihood Estimation (MLE)最大似然估计:假设一个Gaussian噪声$\epsilon$,即$t = y(\mathbf{x}, \mathbf{w}) + /epsilon$,则在训练数据样本i.d.d. 的前提下,对数似然为:
    \[ \ln p(\mathbf{t} | \mathbf{X}, \mathbf{w}, \beta) = \sum_{n=1}^{N} \ln \mathcal{N}( t_n | \mathbf{w}^T \phi(\mathbf{x}_n), \beta^{-1} ) =\
    \frac{N}{2} \ln \beta - \frac{N}{2} \ln (2 \pi) - \beta E_D(\mathbf{w})\]
    其中的$ E_D(\mathbf{w}) = \frac{1}{2} \sum_{n=1}^{N}\{t_n - \mathbf{w}^T \phi(\mathbf{x}_n)\} ^ 2 $就是最小二乘法的目标函数,二者最终殊途同归,即:   
    \textcolor[rgb]{0.00,0.07,1.00}{最小二乘法 $\Longleftrightarrow$ 高斯噪声假设下的MLE}。
    
    
    通过令梯度为0可以得出参数$\mathbf{w}$的解:\[ \mathbf{w}_{ML} = (\Phi^T \Phi)^{-1} \Phi^T \mathbf{t} \],其中$\Phi$
    
    
\end{compactitem}


%\bibliographystyle{abbrvnat} % compatible with the package: Natbib, other choices: plainnat / unsrtnat
%\bibliography{refs}
\end{document} 